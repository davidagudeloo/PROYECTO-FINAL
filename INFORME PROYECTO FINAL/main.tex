\documentclass{article}
\usepackage[utf8]{inputenc}
\usepackage[spanish]{babel}
\usepackage{listings}
\usepackage{graphicx}
\usepackage{geometry}
\graphicspath{ {images/} }
\usepackage{cite}

\geometry{
textheight=23cm
}
\begin{document}

\begin{titlepage}
    \begin{center}
        \vspace*{0cm}
            
        \large
        \textbf{ARTIC WOLF: SAVING THE FAMILY}
            
        \vfill
            
        \vspace{0.8cm}
            
        \textbf{David Agudelo Ochoa}
        
        \vspace{5mm}
        
        \textbf{Erika Dayana León Quiroga}
            
        \vfill
            
        \vspace{0.8cm}
            
        \Large
        Universidad de Antioquia\\
        Despartamento de Ingeniería Electrónica y Telecomunicaciones\\
        Informática II\\
        Medellín-Antioquia\\
        Abril de 2021
            
    \end{center}
\end{titlepage}

\tableofcontents
\newpage
\section{Sección introductoria.}
En este informe veremos el desarrollo del proyecto final de la materia Informática II, el cual será un juego llamado Gartic Wolf: Saving the family. Dentro de las secciones veremos listas de tareas hechas para organizar mejor el trabajo a realizar, actualizaciones y problemas del código del programa.

\section{Lista de tareas.}
\begin{itemize}
    \item Jueves 7 de Abril:
    \begin{itemize}
        \item Crear proyecto Qt
        \item Crear la clase ObjetoAnimado y las que heredarán de ella.
        \item Objeto animado: Atributos y métodos.
        \item Método StringPath.
        \item Probar método animar
        \item Instalar librería QMediaPlayer (para sonidos del juego.)
        
    \end{itemize}
    \item Viernes 8 de Abril:
    \begin{itemize}
        \item .
        \item .
    \end{itemize}

\end{itemize}


    




\section{Conclusiones.}
    \begin{itemize}
        \item
    \end{itemize}

\end{document}
